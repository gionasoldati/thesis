\chapter{Abstract}
\label{chpr:abstract}
Nowadays in Bitcoin it is used ECDSA as underlying digital signature scheme: its drawbacks are well known, ranging from modular inversion to malleability. Schnorr signature, although lacking a standardization, fix these and other problems. Starting from the mathematical and cryptographic foundations, we will follow the recent BIP by Pieter Wuille about a Bitcoin standardization, delving into the technicalities of the comparison between ECDSA and Schnorr, that turns out to be pitiless.

\bigskip
\noindent
Schnorr's linearity property allows many higher level constructions: in this thesis we will focus in particular on the improvements of utilities already implemented in Bitcoin. We will study Schnorr multi-signatures, threshold signatures and the benefits that the new signature scheme could bring to cross-chain atomic swaps and to the Lightning Network.
