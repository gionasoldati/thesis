\chapter{Conclusions}
\label{chpr:conclusion}
The aim of this thesis is to give an introduction to the Schnorr signature algorithm, starting from the mathematics and the cryptography behind the scheme, and present some of its amazing applications to Bitcoin, detailing the benefits and the improvements that would arise from its deployment. We started with a brief but thorough description of the mathematical structures (Chapter \ref{chpr:math}) and cryptographic primitives (Chapter \ref{chpr:ecc}) that underpin digital signature schemes. In Chapter \ref{chpr:dss} we have presented both ECDSA and Schnorr algorithm, respectively the one actually implemented in Bitcoin and the one that is under development. We compared the two schemes, investigating ECDSA lacks and Schnorr benefits, that ranged from security to efficiency. In particular we focused on the linearity property, that turned out to be the key for the higher level construction presented in Chapter \ref{chpr:application}.
\\
We have seen how to traduce utilities already implemented in Bitcoin in terms of Schnorr signatures: multi-signature schemes are implemented through MuSig (Section \ref{musig}), whose main advantage is to recover key aggregation; threshold signatures can be deployed through the protocols presented in Section \ref{threshold}, that makes them indistinguishable from a single signature; the last application we studied has been adaptor signatures and their benefits to cross-chain atomic swaps and to the Lightning Network.

\bigskip
\noindent
The immediate benefits that Schnorr would bring to Bitcoin are improved efficiency and privacy (multi-signatures and threshold signatures would be indistinguishable from a single signature), leading also in an enhancement in fungibility. All this applications would be possible in a straightforward way after the introduction of Schnorr, that could be brought to Bitcoin through a soft-fork\footnote{Improvements in the protocol have to be made without consesus split.}: the fact that Schnorr is superior to ECDSA in every aspect hopefully would ease the process. We have also discussed the possibility of cross-input aggregation, a property that would allow a single signature per transaction: through a proper change in Bitcoin scripting language even this procedure would be possible through a soft fork.
\\
The last thing we would like to point out is that, by no means, the applications presented in this thesis are the unique benefits that Schnorr could bring to Bitcoin. More complex ideas take the names of Taproot (\cite{Taproot}) and Graftroot (\cite{Graftroot}), and are built on top of the concept of MAST an Pay-to-Contract: through these constructions all possible redeem scripts would look the same as a single signer transaction in the cooperative case.